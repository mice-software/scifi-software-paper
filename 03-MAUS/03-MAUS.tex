\section{The MAUS framework}
\label{sec:MAUS}
The tracker software is part of the MICE software framework, known as MAUS (MICE Analysis User Software)~\cite{MausIPAC11}. MAUS is used to perform Monte Carlo simulation and both online and offline data reconstruction. It is built using a combination of C++ and Python, with C++ being used for more processor-intensive tasks and Python being used more in the code presented to the user.  Simulation is based on GEANT4~\cite{GEANT4}, with analysis based on ROOT~\cite{ROOT}.  ROOT files are used as the primary output data format. %and the custom binary format written by the MICE data acquisition system (DAQ) is the primary input. 

MAUS programmes are defined in a Python script together with a configuration file.  This script allows the user to create programmes by combining different MAUS modules depending on the task at hand, following the Map-Reduce programming model~\cite{MapReduce}. The object passed between the modules is known as a ``spill'', representing the data associated with one spill of particles passing through the MICE beamline (see~\cite{MiceBeamline}).  The modules come in four types: Input; Output; Map; and Reduce.  Input modules provide the initial data to MAUS, from a data file, or from the DAQ. Maps perform most of the simulation and analysis work and may be processed in parallel across multiple nodes.  Reducers are used to display output, such as for online reconstruction plots, and are capable of accumulating data sent from maps over multiple spills, but must be run in a single thread. Output modules provide data persistency.

The tracker software consists of 7 maps and a reducer. The maps cover: digitisation of Monte Carlo data; digitisation of real DAQ data; the addition of noise to Monte Carlo data; cluster reconstruction; spacepoint reconstruction; pattern recognition; and the final track fit. The reducer provides real time information on the tracker performance.  %The modules contain little code themselves but instead call C++ classes to perform the work.