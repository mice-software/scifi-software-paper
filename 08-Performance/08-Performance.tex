\section{Performance}
\label{sec:Performance}

  A Monte Carlo simulation was used to determine the efficiency and resolution of the reconstruction algorithms. It was necessarily Monte Carlo based in order to compare the reconstructed events to the simulated truth, thereby highlighting any inefficiencies and inaccuracies.

  The fitted transverse positions, $(x,y)$, and momenta, $(p_x, p_y, p_z)$, were compared to the Monte Carlo truth on an event-by-event basis. The reconstruction of the Monte Carlo data followed the same requirements that are used for the real-data reconstruction. The Monte Carlo truth data was stored at every tracker plane (via virtual hits) to permit a direct comparison with the reconstructed data. All comparisons were made at the tracker reference surface. %the designated measurement location with the MICE cooling channel.

  An artificial beam was generated with uniform distributions for both the longitudinal and transverse momenta. This ensured that the results were not biased by the incoming beam distribution and that the full reconstructible phase-space was probed with equal statistics. In order to remove unwanted non-physical particles, a cut was placed on the Monte Carlo ``truth'' data during the analysis phase to eliminate tracks with a large $p_t/p_z$ ratio. Tracks with a ratio greater than $0.5$ (that is, a $45^\circ$ angle with respect to the $z$ axis) were rejected from the analysis.
  
  \subsection{Kuno's Conjecture}
  \label{sec:performance:kunos_conjecture}
  
  All clusters selected to form triplet spacepoints should follow Kuno's conjecture (section~\ref{subsec:SpacepointReconstruction}). A plot showing the sum of the channel numbers for the clusters in each spacepoint is shown in figure~\ref{fig:kuno}. As expected the sum of the channel numbers for all the clusters in a spacepoint is constant to very good approximation.  The small variation arises from the fact that in real data the channels overlap by approx $1/11^{\mathrm{th}}$, however, in the Monte Carlo digitiser a $50\%$ overlap is assumed. This effect is only seen when there is at least one 2-digit cluster. When only 1-digit clusters are allowed the discrepancy is removed completely.

  \subsection{Track Finding Efficiency}
  \label{sec:performance:track_finding}

  For every simulated event, the number of tracks expected was calculated from the Monte Carlo truth. If a simulated track crossed enough tracker planes to create a sufficient number of spacepoints (3 for straight tracks and 4 for helical tracks), a reconstructed track was expected. The parameters of the reconstructed tracks were compared to the expected track parameters. The efficiency of track finding as a function of the true longitudinal and transverse momentum is shown in figure~\ref{fig:track_efficiency}. High efficiency is observed across the whole space with the expection of the low transverse momentum region where a reduction is observed. This effect is believed to be an artifact of the pattern recognition track model, as the helix radius tends to infinity as the transverse momentum approaches zero. This effect is enhanced by high longitudinal momentum, which will reduce the track curvature within the tracker volume still further (visible as a small reduction in the efficiency in the low transverse, high longitudinal momentum region). This effect is the subject of ongoing optimisation work.

  %Once the the Monte Carlo tracks have been identified, the expected number of trackpoints can be calculated by examining the number of tracker planes that the simulated track crossed. Comparing the number of trackpoints in each reconstructed track to the expected number for each simulated track permits the efficiency of finding the correct number of trackpoints to be calculated. Figure~\ref{fig:tp_efficiency} shows the trackpoint finding efficiency as a function of longitudinal and transverse momenta.

  \subsection{Position and Momentum Resolution}
  \label{sec:performance:resolutions}
  
  %Position residuals are shown in figures~\ref{fig:XResidKalman} and \ref{fig:YResidKalman}, and the momentum residuals in figures~\ref{fig:PtResidKalman} and \ref{fig:PzResidKalman}.  The position reconstruction can be seen to agree with the Monte Carlo truth to high precision in both the upstream and downstream trackers. The momentum residuals currently display some systematic effect which is due to uncertainty in the determination of energy loss within the tracker and the accuracy of the pattern recognition parameters.
  
  The $\chi^2$ per degree of freedom for the track fits, used to provide the position and momentum determinations, are shown in figure~\ref{fig:track_chisq}. 
  The position residuals, shown in figures~\ref{fig:XResidKalman} and \ref{fig:YResidKalman}, are consistent with the expected measurement resolutions for a combined fit. The transverse momentum resolution, shown in figure~\ref{fig:PtResidKalman}, is consistent across the range of the sensitive phase-space at $\sim$0.9~MeV/c in both trackers. The longitudinal momentum shown in figure~\ref{fig:PzResidKalman}, an intrinsically more difficult measurement for the tracker, retains an acceptable spread of ${\sim4}$~MeV/c in both trackers. There is however a small systematic effect visible in the momentum distributions.

  In order to produce these plots, a requirement that there was a cluster within the reference plane was applied. Due to the effects of Multiple Coulomb Scattering, on rare occasions a single hard scatter can cause pattern recognition to miss a single spacepoint at the reference plane, hence creating a tail that will adversely affect the distributions. As we are concerned with the resolution following a successful pattern recognition stage, these events were removed.
  
  The track fit assumes a simple model of energy loss based on the mean thickness of materials used in the tracker construction, however the effects of glues and resins, in addition to the non-uniform densities of the materials are not considered. This results in a systematic underestimate of the energy lost at each tracker plane and a small systematic bias during the pattern recognition stage, which doesn't attempt to model these effects. Due to the high-granularity of the position information used by the fit, the position reconstruction is much less sensitive to any discrepancies in momentum. %Figure~\ref{fig:pBiasKalman} shows the mean deviation in total reconstructed momentum across the typical momentum acceptance of the trackers.
 
  Trends in transverse and longitudinal momentum resolution as a function of transverse momentum are shown in figures~\ref{fig:PtPtResolKalman} and \ref{fig:PtPzResolKalman}. A consistently uniform distribution is found in the transverse momentum as expected, with the predominant issue found in the longitudinal reconstruction of low-$p_t$ tracks. This effect, when coupled with variations in efficiency will yield some systematic concerns in the reconstruction of statistical quantities such as emittance. Studies of these systematic biases are currently under way.

 

