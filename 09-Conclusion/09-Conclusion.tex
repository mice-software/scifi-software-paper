\section{Conclusion}
\label{sec:Conclusion}
The tracker software has been presented and the performance of the reconstruction algorithms, including: spacepoint reconstruction; pattern recognition; and the final track fit, has been shown to meet expectation.

The performance of the final track fit was evaluated by comparing Monte Carlo truth with reconstructed data, for the key tracker measurements of $x$, $y$, $p_{t}$ and $p_z$.  The observed performance in the transverse position measurements is excellent, for both the upstream and downstream trackers. The reconstruction resolution in $p_t$ and $p_z$ meet specification and will provide a sufficient degree of precision for the MICE physics program. 

The track model uses a simplification of the energy loss in the tracker planes which leads to the systematic shift discussed in section 8.3. Since the two trackers were constructed to be identical this shift will cancel in the calculation of energy loss. Monte Carlo studies will allow us to calculate a linear correction to the measured momentum and account for the small remaining shift. We will use measurements by other detectors in MICE to validate this correction.

The tracker software is now used routinely in MICE data reconstruction, providing data for analysis and publication.

% If the approximations in the track model, and their effect on the track fits, limits the ability to measure cooling, the current Kalman filter can be upgraded to an adaptive Kalman filter where the energy loss is both modelled and estimated by the track fit. The fitter was written to accept such upgrades.
% 
% MICE will measure the transverse emittance of a muon beam which requires the determination of the beam covariance matrix. Our errors will be small and symmetric which will permit a simple covariance matrix correction to be applied and allow an accurate and reliable measurement of muon beam cooling. The tracker software is now used routinely in MICE data reconstruction, providing data for analysis and publication.

% There are, however, systematic effects present in the momentum reconstruction due to the complexity of modelling the density and thicknesses of the various materials within the tracker. The Monte Carlo model provides a more detailed description than could feasibly be implemented in the reconstruction, hence producing the residuals seen in section~\ref{sec:performance:resolutions}. This discrepancy will be representative of a physical systematic effect in the reconstruction of data and corresponds to the leading systematic effect in the tracker reconstruction.
% 
% Monte Carlo studies will be used to model the momentum discrepancy and provide a linear correction. The linear correction can then be used during analyses to reduce the effect of this systematic. Comparisons with other detectors in the MICE experiment may provide a data-driven estimate for this correction thereby supporting the Monte Carlo model. In addition it would be possible to further extend the track fit such that an Adaptive Kalman Filter is used, whereby the energy loss per plane is both modelled and estimated by the track fit. Such extensions are currently under discussion.
% 
% Due to the precision of the track fit the key measurement of MICE, the transverse emittance of a muon beam, meets specification. The emittance is based on the precise determination of the beam covariance matrix, and is therefore sensitive to the resolution of the reconstruction. The resolutions of the individual parameters are small and symmetric which permits a simple covariance matrix correction to be applied, fully accounting for the measurement effects of the tracker. Additionally, the high resolution reduces the number of muons required to acheive the expected statistical precision of MICE.

% The performance of the final Kalman filter-based track fit has been evaluated by comparing Monte Carlo truth with reconstructed data, for the key tracker measurements of $x$, $y$, $p_{t}$ and $p_z$.  The observed performance in the transverse position is excellent for both the upstream and downstream trackers. The reconstruction resolution in $p_t$ and $p_z$ meet specification and will provide a sufficient degree of precision for the MICE physics program.

%There are, however, systematic effects present in the momentum reconstruction due to the complexity of modelling the density and thicknesses of the various materials within the tracker. The Monte Carlo model provides a more detailed description than could feasibly be implemented in the reconstruction, hence producing the residuals seen in section~\ref{sec:performance:resolutions}. This discrepancy will be representative of a physical systematic effect in the reconstruction of data and corresponds to the leading systematic effect in the tracker reconstruction.

% Systematic effects are present in the momentum reconstruction, which have the potential to produce a corresponding systematic effect in the final emittance measurement unless accounted for. Monte Carlo studies will be used to model the momentum discrepancy and provide a linear correction. The linear correction can then be used during analyses to reduce the effect of this systematic. Comparisons with other detectors in the MICE experiment may provide a data-driven estimate for this correction thereby supporting the Monte Carlo model. In addition it would be possible to further extend the track fit such that an Adaptive Kalman Filter is used, whereby the energy loss per plane is both modelled and estimated by the track fit. Such extensions are currently under discussion.

%Due to the precision of the track fit the key measurement of MICE, the transverse emittance of a muon beam, meets specification. The emittance is based on the precise determination of the beam covariance matrix, and is therefore sensitive to the resolution of the reconstruction. The resolutions of the individual parameters are small and symmetric which permits a simple covariance matrix correction to be applied, fully accounting for the measurement effects of the tracker. Additionally, the high resolution reduces the number of muons required to acheive the expected statistical precision of MICE.